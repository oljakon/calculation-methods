\documentclass[a4paper,14pt]{article}

\usepackage{amsmath}
\usepackage{cmap} 
\usepackage[T2A]{fontenc} 
\usepackage[utf8]{inputenc} 
\usepackage[english,russian]{babel} 
\usepackage[left = 2cm, right = 1cm, top = 2cm, bottom = 2 cm]{geometry}
\usepackage{listings}
\usepackage{graphicx}
\usepackage{indentfirst}
\usepackage{color} 
\usepackage{float} 
\usepackage{multirow}
\usepackage{amsmath, amsfonts, amssymb, mathtools} 

\usepackage{titlesec, blindtext, color}


\lstset{
	language=Matlab, 
	numbers=left,   
	frame=single, 
	escapebegin=\begin{russian}\commentfont,
    escapeend=\end{russian},
    breaklines=true,   
    breakatwhitespace=true,
    showstringspaces=false,
	literate={а}{{\selectfont\char224}}1
	{б}{{\selectfont\char225}}1
	{в}{{\selectfont\char226}}1
	{г}{{\selectfont\char227}}1
	{д}{{\selectfont\char228}}1
	{е}{{\selectfont\char229}}1
	{ё}{{\"e}}1
	{ж}{{\selectfont\char230}}1
	{з}{{\selectfont\char231}}1
	{и}{{\selectfont\char232}}1
	{й}{{\selectfont\char233}}1
	{к}{{\selectfont\char234}}1
	{л}{{\selectfont\char235}}1
	{м}{{\selectfont\char236}}1
	{н}{{\selectfont\char237}}1
	{о}{{\selectfont\char238}}1
	{п}{{\selectfont\char239}}1
	{р}{{\selectfont\char240}}1
	{с}{{\selectfont\char241}}1
	{т}{{\selectfont\char242}}1
	{у}{{\selectfont\char243}}1
	{ф}{{\selectfont\char244}}1
	{х}{{\selectfont\char245}}1
	{ц}{{\selectfont\char246}}1
	{ч}{{\selectfont\char247}}1
	{ш}{{\selectfont\char248}}1
	{щ}{{\selectfont\char249}}1
	{ъ}{{\selectfont\char250}}1
	{ы}{{\selectfont\char251}}1
	{ь}{{\selectfont\char252}}1
	{э}{{\selectfont\char253}}1
	{ю}{{\selectfont\char254}}1
	{я}{{\selectfont\char255}}1
	{А}{{\selectfont\char192}}1
	{Б}{{\selectfont\char193}}1
	{В}{{\selectfont\char194}}1
	{Г}{{\selectfont\char195}}1
	{Д}{{\selectfont\char196}}1
	{Е}{{\selectfont\char197}}1
	{Ё}{{\"E}}1
	{Ж}{{\selectfont\char198}}1
	{З}{{\selectfont\char199}}1
	{И}{{\selectfont\char200}}1
	{Й}{{\selectfont\char201}}1
	{К}{{\selectfont\char202}}1
	{Л}{{\selectfont\char203}}1
	{М}{{\selectfont\char204}}1
	{Н}{{\selectfont\char205}}1
	{О}{{\selectfont\char206}}1
	{П}{{\selectfont\char207}}1
	{Р}{{\selectfont\char208}}1
	{С}{{\selectfont\char209}}1
	{Т}{{\selectfont\char210}}1
	{У}{{\selectfont\char211}}1
	{Ф}{{\selectfont\char212}}1
	{Х}{{\selectfont\char213}}1
	{Ц}{{\selectfont\char214}}1
	{Ч}{{\selectfont\char215}}1
	{Ш}{{\selectfont\char216}}1
	{Щ}{{\selectfont\char217}}1
	{Ъ}{{\selectfont\char218}}1
	{Ы}{{\selectfont\char219}}1
	{Ь}{{\selectfont\char220}}1
	{Э}{{\selectfont\char221}}1
	{Ю}{{\selectfont\char222}}1
	{Я}{{\selectfont\char223}}1
}


\begin{document}
\begin{titlepage}

    \begin{table}[H]
        \centering
        \footnotesize
        \begin{tabular}{cc}
            \multirow{8}{*}{\includegraphics[scale=0.35]{bmstu}}
            & \\
            & \\
            & \textbf{Министерство науки и высшего образования Российской Федерации} \\
            & \textbf{Федеральное государственное бюджетное образовательное учреждение} \\
            & \textbf{высшего образования} \\
            & \textbf{<<Московский государственный технический} \\
            & \textbf{университет имени Н.Э. Баумана>>} \\
            & \textbf{(МГТУ им. Н.Э. Баумана)} \\
        \end{tabular}
    \end{table}

    \begin{flushleft}
        \rule[-1cm]{\textwidth}{3pt}
        \rule{\textwidth}{1pt}
    \end{flushleft}

    \begin{flushleft}
        \small
        ФАКУЛЬТЕТ
        \underline{<<Информатика и системы управления>>\ \ \ \ \ \ \ 
        \ \ \ \ \ \ \ \ \ \ \ \ \ \ \ \ \ \ \ \ \ \ \ \ \ \ \ \ \ \ \ 
    \ \ \ \ \ \ \ \ \ \ \ \ \ \ \ } \\
        КАФЕДРА
        \underline{<<Программное обеспечение ЭВМ и
        информационные технологии>>
        \ \ \ \ \ \ \ \ \ \ \ \ \ \ \ \ \ \ \ \ }
    \end{flushleft}

    \vspace{4cm}

     \begin{center}
        \textbf{Лабораторная работа № 1} \\
        \textbf{По курсу <<Методы вычислений>>.} \\
        \vspace{0.5cm}
        \textbf{Венгерский метод решения задачи о назначениях}
    \end{center}

    \vspace{3cm}

    \begin{flushleft}
        \begin{tabular}{ll}
            \textbf{Студент} & Кондрашова О.П. \\
            \textbf{Группа} & ИУ7-11М \\
             \textbf{Вариант} & 13 \\
            \textbf{Преподаватель} & Власов П.А. \\
        \end{tabular}
    \end{flushleft}

    \vspace{7cm}

   \begin{center}
        Москва, 2021 г.
    \end{center}

\end{titlepage}


\textbf{Цель работы: } изучение венгерского метода решения задачи о назначениях.

\textbf{Содержание работы}
\begin{enumerate}
\item реализовать венгерский метод решения задачи о назначениях в виде программы на ЭВМ;
\item провести решение задачи с матрицей стоимостей, заданной в индивидуальном варианте, рас-
смотрев два случая:
\begin{enumerate}
\item задача о назначениях является задачей минимизации,
\item задача о назначениях является задачей максимизации.
\end{enumerate}
\end{enumerate}

\section{Теоретическая часть}

\subsection{Содержательная постановка задачи}

В распоряжении работодателя имеется $n$ работ и $n$ исполнителей. Стоимость выполнения $i$-ой работы $j$-ым исполнителем составляет $c_{ij} \geq 0$ единиц. Требуется распределить работу между исполнителями так, чтобы:
\begin{enumerate}
\item каждый исполнитель выполнял ровно одну работу;
\item общая стоимость выполнения всех работ была минимальной.
\end{enumerate}

\subsection{Математическая постановка задачи}

Обозначим $C=(c_{ij})_{i,j = \overline{1;n}}$, где $C$ -- матрица стоимостей.
Введем управляемые переменные:

\begin{equation}
x_{ij}(x) =
 \begin{cases}
   1, &\text{если $i$-ую работу выполняет $j$-ый работник}
   \\
   0, &\text{иначе}
 \end{cases},  
\end{equation}
где $i,j = \overline{1,n}$\\

Обозначим $X=(x_{ij})_{i,j = \overline{1;n}}$, где $X$ -- матрица назначений.

Тогда общая стоимость выполнения работ:

\begin{equation}
f=\sum\limits_{i=1}^n\sum\limits_{j=1}^n c_{ij} \cdot x_{ij}
\end{equation}

Условие того, что $j$-й работник выполняет ровно одну работу:

\begin{equation}
\sum\limits_{i=1}^n x_{ij} = 1,
\end{equation}
где $j = \overline{1;n}$

Условие того, что $i$-ю работу выполняет один работник:

\begin{equation}
\sum\limits_{j=1}^n x_{ij} = 1,
\end{equation}
где $i = \overline{1;n}$

Тогда математическая постановка задачи о назначениях имеет следующий вид:

\begin{equation}
x_{ij}(x) =
 \begin{cases}
   f=\sum\limits_{i=1}^n\sum\limits_{j=1}^n c_{ij} \cdot x_{ij} \rightarrow \min
   \\
   \sum\limits_{i=1}^n x_{ij} = 1, \quad j = \overline{1;n}
   \\
   \sum\limits_{j=1}^n x_{ij} = 1, \quad i = \overline{1;n}
   \\
   x_{i,j}\subseteq{0,1}, \quad i,j = \overline{1;n}
 \end{cases},  
\end{equation}
где $f$ -- целевая функция, $c_{ij}$ -- элементы матрицы стоимостей $C$, $x_{ij}$ -- элементы матрицы назначений $X$.

\subsection{Исходные данные индивидуального варианта}

\begin{equation}
\begin{bmatrix}
10 & 4 & 9 & 8 & 5\\
9 & 3 & 5 & 7 & 8\\
2 & 5 & 8 & 10 & 5\\
4 & 5 & 7 & 9 & 3\\
8 & 7 & 10 & 9 & 6
\end{bmatrix}
\end{equation}


\subsection{Венгерский метод}

Венгерский метод решения задачи о назначениях применяется для решения задачи минимизации, то есть для случая, когда матрица стоимостей $C$ содержит в себе стоимости работ. Также матрица стоимостей $C$ может использоваться для решения задачи максимизации, в этом случае ее элементы отражают прибыль работодателя от назначения $j$-го работника на $i$-ю работу. 

Решение задачи максимизации отличается от решения задачи минимизации тем, что в данном случае в начале алгоритма добавляется шаг, на котором необходимо найти максимальный элемент столбца, вычесть его из каждого элемента матрицы и домножить матрицу на $-1$.

Если решается задача минимизации, алгоритм начинается со следующих шагов:
\begin{enumerate}
\item в каждом столбце матрицы стоимостей выбирается наименьший элемент и вычитается из всех элементов этого столбца;
\item в каждой строке полученной матрицы выбирается наименьший элемент и вычитается из всех элементов этой строки.
\end{enumerate}

В результате получается матрица, в каждой строке и столбце которой присутствует хотя бы один нуль.

Системой независимых нулей (СНН) будем называть такой набор нулей матрицы стоимостей, в котором никакие два элемента не стоят ни в общей строке, ни в общем столбце.

Для построения начальной СНН необходимо просмотреть столбцы текущей матрицы с размерностью $n \times n$ и первый найденный в столбце 0, в одной строке с которым нет элемента $0^*$, отметить как $0^*$. 

Если полученная СНН содержит $n$ элементов, значит, найдено оптимальное решение. Иначе СНН необходимо улучшить.

Для улучшения СНН необходимо (первый способ):
\begin{enumerate}
\item отметить + столбцы, в которых стоят $0^*$ (данные столбцы и все их элементы будем называть выделенными);
\item найти 0 среди невыделенных элементов и отметить его следующим образом: $0'$;
\item если в одной строке с $0'$ стоит $0^*$, снять выеделение со столбца с найденным $0^*$ и отметить строку с текущим $0'$, затем повторить поиск;

\item если в строке с выделенным $0'$ нет эелемента $0^*$, то строится L-цепочка:

$текущий 0' \rightarrow \text{по столбцу} \rightarrow 0^* \rightarrow \text{по строке} \rightarrow 0' \rightarrow ... \rightarrow 0'$

В пределах цепочки заменить элементы: $0^* \rightarrow 0, 0' \rightarrow 0^*$
\end{enumerate}

Если среди невыделенных элементов не найдено нулей, то необходимо (второй способ):
\begin{enumerate}
\item среди всех невыделенных элементов выбрать наименьший элемент $h>0$;
\item вычесть h из элементов в невыделенных столбцах;
\item прибавить h к элементам в выделенных строках.
\end{enumerate}

В результате преобразований получим эквивалентную матрицу, среди невыделенных элементов которой есть 0, после этого можно повторить преобразования первого способа.

На рис. \ref{schema} представлена схема алгоритма венгерского метода решения задачи о назначениях для задачи минимизации.

\begin{figure}[H]
	\begin{center}
    	\includegraphics[scale=0.42]{schema2}
        \caption{Результат работы программы}
        \label{schema}
	\end{center}
\end{figure}

\section{Практическая часть}

\subsection{Листинг программы}

В листинге \ref{some-code} приведен текст программы.

\begin{lstlisting}[label=some-code,caption=Листинг программы]
function lab1()
C = [10, 4, 9, 8, 5;
    9, 3, 5, 7, 8;
    2, 5, 8, 10, 5;
    4, 5, 7, 9, 3;
    8, 7, 10, 9, 6];

debug = true;
%debug = false;
task_max = true;
%task_max = false;

[rows, cols] = size(C);

C1 = C;

if task_max
    % Поиск максимального элемента, вычитание его из элементов
    % матрицы и домножение на -1
    max_cost = 0;
    for i = 1:rows
        for j = 1:cols
            if C(i,j) > max_cost
                max_cost = C(i,j);
            end
        end
    end
    temp_c = C;
    for i = 1:rows
        for j = 1:cols
            C1(i,j) = temp_c(i,j)*(-1) + max_cost;
        end
    end
    
    if debug
        disp("Матрица после первого шага максимизации");
        disp(C1);
    end
end

% Поиск наименьшего элемента столбца и его вычитание из элементов столбца
min_c = min(C1, [], 1);
C1 = C1 - min_c;
if debug
    disp("Матрица после вычитания наименьшего элемента по столбцам");
    disp(C1);
end

% Поиск наименьшего элемента строки и его вычитание из элементов строки
min_r = min(C1, [], 2);
C1 = C1 - min_r;
if debug
    disp("Матрица после вычитания наименьшего элемента по строкам");
    disp(C1);
end

% Первичный поиск 0*
stars = find_stars(C1);

N = size(C1, 1);
quotes = zeros(N);

if debug
    disp("Выделенные 0* для определения CHH");
    debug_matrix(C1, stars, quotes, [], zeros(N));
end
% |CHH|
CNN = nnz(stars);
if debug
    disp("CHH = ");
    disp(CNN);
end
if CNN < cols
    if debug
        disp("СНН нужно улучшать");
    end
    
    cur_row = 0;
    cur_col = 0;
    
    iteration = 1;
    while CNN < rows
        if debug
            disp("Номер итерации: ");
            disp(iteration);
        end

        markedCol = find_marked_cols(stars);
        markedRows = zeros(size(C1,2));
        while true
            % Поиск неотмеченных нулей
            [succ, cur_col, cur_row] = find_unm_zeros(C1, markedCol, markedRows);
            % Если 0 найден
            if succ
                quotes(cur_row,cur_col) = 1;
                if debug
                    disp("Матрица после нахождения неотмеченного нуля");
                    debug_matrix(C1, stars, quotes, markedCol, markedRows);
                end
                % Отметить и продолжить поиск
                [res, col] = check_marked_zero_in_row(cur_row, stars);
                if res == true
                    markedRows(cur_row) = 1;
                    markedCol(col) = 0;
                    if debug
                        disp("Матрица после переопределения выделений строк и столбцов");
                        debug_matrix(C1, stars, quotes, markedCol, markedRows);
                    end
                % Если ноль не найден, построить L-цепочку
                else
                    stars = create_L_chain(stars, quotes, cur_row, cur_col);
                    CNN=CNN+1;
                    markedCol = find_marked_cols(stars);
                    markedRows = zeros(size(C1,2));
                    if debug
                        disp("Матрица 0* после построения L-цепочки: ");
                        disp(stars);
                    end 
                    break;
                end
            % Если ноль не найден, вычесть h
            else
                C1 = calc_h(C1, markedRows, markedCol);
                if debug
                    disp("Матрица после вычитания h из столбцов и прибавления к строкам");
                    debug_matrix(C1, stars, quotes, markedCol, markedRows);
                end
            end
        end
    iteration = iteration + 1;
    end
end

% X_opt и f_opt
X_opt = zeros(cols);
for i = 1:rows
    for j = 1:cols
        if stars(i,j)
            X_opt(i,j) = 1;
        end
    end
end
disp("X_opt = ");
disp(X_opt);
f_opt = 0;
for i = 1:rows
    for j = 1:cols
        if stars(i,j)
            f_opt = f_opt + C(i,j)*stars(i,j);
        end
    end
end
disp("f_opt = ");
disp(f_opt);

% Функция первичного поиска 0*
function res = find_stars(C1)
    N = size(C1,1);
    stars = zeros(N);
    for i = 1:N
        for j = 1:N
            if C1(i,j) == 0
                if max(stars(:,j))==0 && max(stars(i,:))==0
                    stars(i,j) = 1;
                    break;
                end
            end
        end
    end
    res = stars;
end

% Функция поиска отмеченных столбцов
function marked_cols = find_marked_cols(stars)
    marked_cols = sum(stars);
end

% Функция поиска неотмеченного нуля в матрице
function [res,col,row] = find_unm_zeros(C1, markedCol, markedRows)
    N = size(C1,1);
    col = 0;
    row = 0;
    res = false;
    for j = 1:N
        for i = 1:N
            if C1(i,j) == 0 && markedCol(j) == 0 && markedRows(i) == 0
                res = true;
                col = j;
                row = i;
                break;
            end
        end
    end
end

% Функция, которая определяет, есть ли в строке матрицы 0*
function [res, col] = check_marked_zero_in_row(row, stars)
    res = false;
    col = 0;
    for j = 1:size(stars, 2)
        if stars(row, j) == 1
            res = true;
            col = j;
            break
        end
    end
end

% Функция поиска 0* в столбце
function [res, row] = check_marked_zero_in_column(col,stars)
    [res,row] = check_marked_zero_in_row(col,stars');
end

% Функция пересчета матрицы на основе поиска h и его вычитания из невыделенных столбцов
function matrix = calc_h(C1, markedRows, markedCol)
    matrix = C1;
    N = size(C1,1);
    min_el = -1;
    
    % Поиск минимального элемента из невыделенных
    for j = 1:N
        for i = 1:N
            if markedCol(j) == 0 && markedRows(i) == 0
                if matrix(i, j) < min_el || min_el == -1
                    min_el = matrix(i, j);
                end
            end
        end
    end
    
    % Вычитание минимального элемента из невыделенных
    for j = 1:N
        for i = 1:N
            if markedCol(j) == 0 && markedRows(i) == 0
                matrix(i,j) = matrix(i,j) - min_el;
            end
            if markedCol(j) == 1 && markedRows(i) == 1
                matrix(i,j) = matrix(i,j) + min_el;
            end
        end
    end
end

% Функция построения L-цепочки
function created_L = create_L_chain(stars, quotes, row, col)
    res = true;
    cur_col = col;
    cur_row = row;
    created_L = stars;
    while res
        created_L(cur_row,cur_col) = 1;
        [res,cur_row] = check_marked_zero_in_column(cur_col,stars);
        if(res == true)
            stars(cur_row, cur_col) = 0;
            created_L(cur_row, cur_col) = 0;
            [res, cur_col] = check_marked_zero_in_row(cur_row,quotes);
        end
    end
end

% Функция вывода матрицы с 0*, 0'
function debug_matrix(C1, stars, quotes, markedCol, markedRows)
    for i = 1:size(C1,1)
        for j = 1:size(C1,2)
            if stars(i,j) == 1
                fprintf("%d* \t", C1(i,j));
            else
                if quotes(i,j) == 1
                    fprintf("%d' \t", C1(i,j));
                else
                    fprintf("%d \t", C1(i,j));
                end
            end
        end
        if markedRows(i)
            fprintf(' +\n');
        else
            fprintf(' \n');
        end 
    end
    for i = 1:length(markedCol)
        if markedCol(i)
            fprintf("+ \t");
        else
            fprintf(" \t");
        end
    end
    fprintf("\n");
end

end
\end{lstlisting}

\subsection{Задача максимизации}

Исходная матрица:

\begin{equation} C=
\begin{bmatrix}
10 & 4 & 9 & 8 & 5\\
9 & 3 & 5 & 7 & 8\\
2 & 5 & 8 & 10 & 5\\
4 & 5 & 7 & 9 & 3\\
8 & 7 & 10 & 9 & 6
\end{bmatrix}
\end{equation}

На первом шаге максимизации определяем максимальный элемент матрицы стоимостей -- 10. Вычитаем найденный максимум из всех элементов матрицы и домножаем их на -1:

\begin{equation}
\begin{bmatrix}
     0  &   6  &   1  &   2  &   5\\
     1  &   7  &   5  &   3  &   2\\
     8  &   5  &   2  &   0  &   5\\
     6  &   5  &   3  &   1  &   7\\
     2  &   3  &   0  &   1  &   4\\
\end{bmatrix}
\end{equation}

Для каждого из столбцов вычтем минимальные элементы:

\begin{equation}
\begin{bmatrix}
     0  &   3  &   1  &   2  &   3\\
     1  &   4  &   5  &   3  &   0\\
     8  &   2  &   2  &   0  &   3\\
     6  &   2  &   3  &   1  &   5\\
     2  &   0  &   0  &   1  &   2\\
\end{bmatrix}
\end{equation}

Для каждой из строк вычтем минимальные элементы:

\begin{equation}
\begin{bmatrix}
     0  &   3  &   1  &   2  &   3\\
     1  &   4  &   5  &   3  &   0\\
     8  &   2  &   2  &   0  &   3\\
     5  &   1  &   2  &   0  &   4\\
     2  &   0  &   0  &   1  &   2\\
\end{bmatrix}
\end{equation}

Выделим 0* и составим СНН:

\begin{equation}
\begin{bmatrix}
0^* &	3 &		1  &	2   &	3 	 \\
1  &	4 &		5  &	3   &	0^*  \\
8  &	2 &		2  &	0^* & 	3 	 \\
5  &    1 &		2  &	0   &	4 	 \\
2  &	0^* &	0' &	1   &	2 	 \\
\end{bmatrix}
\end{equation}
\begin{equation}
\begin{matrix}
+ & + & - & + & +
\end{matrix}
\end{equation}

$|CHH| = 4 < 5 = n$, следовательно, СНН нужно улучшать.

1 итерация.

Матрица после переопределения выделений строк и столбцов:

\begin{equation}
\begin{bmatrix}
0^* &	3 &		1  &	2   &	3 	 \\
1  &	4 &		5  &	3   &	0^*  \\
8  &	2 &		2  &	0^* & 	3 	 \\
5  &    1 &		2  &	0   &	4 	 \\
2  &	0^* &	0' &	1   &	2 	 \\
\end{bmatrix}
\begin{matrix}
- \\
- \\
- \\
- \\
+
\end{matrix}
\end{equation}
\begin{equation}
\begin{matrix}
+ & - & - & + & +
\end{matrix}
\end{equation}

Находим $h=1>0$ -- наименьший элемент в невыделенных столбцах. Вычитаем $h$ из невыделенных столбцов:

\begin{equation}
\begin{bmatrix}
0^* &	2  & 	0   &	2   &	3 	\\ 
1  &	3  & 	4   &	3   &	0^* 	\\
8  &	1  & 	1   &	0^*  & 	3 	\\
5  & 	0  & 	1   &	0   &	4 	\\
2  &   -1^* &  -1'  &	1   &	2 	\\
\end{bmatrix}
\end{equation}

\newpage
Прибавляем $h$ к выделенным строкам:

\begin{equation}
\begin{bmatrix}
0^* &	2  & 	0   &	2   &	3 	\\ 
1  &	3  & 	4   &	3   &	0^* \\
8  &	1  & 	1   &	0^* & 	3 	\\
5  & 	0  & 	1   &	0   &	4 	\\
3  &	0^*  & 	0'  &	2   &	3 	\\
\end{bmatrix}
\end{equation}

Находим неотмеченный 0 и переопределяем выделение столбцов и строк:

\begin{equation}
\begin{bmatrix}
0^* &	2  & 	0'   &	2   &	3 	\\ 
1  &	3  & 	4   &	3   &	0^* \\
8  &	1  & 	1   &	0^* & 	3 	\\
5  & 	0  & 	1   &	0   &	4 	\\
3  &	0^*  & 	0'  &	2   &	3 	\\
\end{bmatrix}
\begin{matrix}
+ \\
- \\
- \\
- \\
+
\end{matrix}
\end{equation}
\begin{equation}
\begin{matrix}
- & - & - & + & +
\end{matrix}
\end{equation}

Построим L-цепочку, начиная с элемента с индексами (4; 2), где 4 -- номер строки, 2 -- номер столбца. Тогда в цепочку так же войдут элементы с индексами (5; 2) и (5; 3).

Матрица после построения L-цепочки и  переопределения отмеченных элементов:

\begin{equation}
\begin{bmatrix}
0^* &	2  & 	0'   &	2   &	3 	\\ 
1  &	3  & 	4   &	3   &	0^* \\
8  &	1  & 	1   &	0^* & 	3 	\\
5  & 	0^* & 	1   &	0   &	4 	\\
3  &	0  & 	0^*  &	2   &	3 	\\
\end{bmatrix}
\end{equation}

$|CHH| = 5 = n$, следовательно, СНН больше улучшать не нужно. 

\begin{equation} X_{opt} = 
\begin{bmatrix}
1  &	0  & 	0   &	0   &	0 	\\ 
0  &	0  & 	0   &	0   &	1   \\
0  &	0  & 	0   &	1   & 	0 	\\
0  & 	1  & 	0   &	0   &	0 	\\
0  &	0  & 	1   &	0   &	0 	\\
\end{bmatrix}
\end{equation}

\begin{equation}
f_{opt} = 10+5+10+10+8=43
\end{equation}

\subsection{Задача минимизациии}

Исходная матрица:

\begin{equation} C=
\begin{bmatrix}
10 & 4 & 9 & 8 & 5\\
9 & 3 & 5 & 7 & 8\\
2 & 5 & 8 & 10 & 5\\
4 & 5 & 7 & 9 & 3\\
8 & 7 & 10 & 9 & 6
\end{bmatrix}
\end{equation}

Для каждого из столбцов вычтем минимальные элементы:

\begin{equation}
\begin{bmatrix}
	 8  &   1  &   4  &   1  &   2\\
     7  &   0  &   0  &   0  &   5\\
     0  &   2  &   3  &   3  &   2\\
     2  &   2  &   2  &   2  &   0\\
     6  &   4  &   5  &   2  &   3\\
\end{bmatrix}
\end{equation}

Для каждой из строк вычтем минимальные элементы:

\begin{equation}
\begin{bmatrix}
     7  &   0  &   3  &   0  &   1\\
     7  &   0  &   0  &   0  &   5\\
     0  &   2  &   3  &   3  &   2\\
     2  &   2  &   2  &   2  &   0\\
     4  &   2  &   3  &   0  &   1\\
\end{bmatrix}
\end{equation}

\newpage
Выделим 0* и составим СНН:

\begin{equation}
\begin{bmatrix}
     7    &   0^*  &   3    &   0    &   1\\
     7    &   0    &   0^*  &   0    &   5\\
     0^*  &   2    &   3    &   3    &   2\\
     2    &   2    &   2    &   2    &   0^*\\
     4    &   2    &   3    &   0^*  &   1\\
\end{bmatrix}
\end{equation}
\begin{equation}
\begin{matrix}
+ & + & - & + & +
\end{matrix}
\end{equation}

$|CHH| = 5 = n$, следовательно, СНН больше улучшать не нужно. 

\begin{equation} X_{opt} = 
\begin{bmatrix}
     0  &   1  &   0  &   0  &   0\\
     0  &   0  &   1  &   0  &   0\\
     1  &   0  &   0  &   0  &   0\\
     0  &   0  &   0  &   0  &   1\\
     0  &   0  &   0  &   1  &   0\\
\end{bmatrix}
\end{equation}

\begin{equation}
f_{opt} = 2+4+5+9+3 = 23
\end{equation}


\section{Заключение}

В результате выполнения лабораторной работы был реализован венгерский метод решения задачи о назначениях для задач минимизации и максимизации. В ходе выполнения лабораторной работы была написана программа на языке MatLab, позволяющая применять венгерский метод решения задачи о назначениях.

\end{document}